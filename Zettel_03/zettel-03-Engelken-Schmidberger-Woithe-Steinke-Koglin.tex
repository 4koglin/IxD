\documentclass[a4paper,10pt]{article}

% Hier die Nummer des Blatts und Autoren angeben.
\newcommand{\blatt}{3}
\newcommand{\autor}{Ralf Engelken, Joachim Schmidberger, Frank Woithe, Michael Steinke, Merlin Koglin}

\usepackage{hci}


\begin{document}
% Seitenkopf mit Informationen
\kopf
\renewcommand{\figurename}{Figure}

\aufgabe{5 \textit{(Team-Aufgabe)}} 
Laut Fitt's Law gilt
\[
	IoD = \log_2{\left(\frac{D}{S} + 1 \right)}
\]
Wobei $D$ die Distanz zum Ziel und $S$ die Breite bzw. Größe des Ziels ist.
\begin{enumerate}
	\item 
		\begin{enumerate}
			\item Pull-Down-Menü
			\begin{enumerate}
				\item Option 1
				 	\[ D =\frac{ \sqrt{5px \cdot 60px}}{2} \]
				 	\[ S = 2D \]
					\[ IoD = \log_2{\left(\frac{D}{2D} + 1 \right) = \log_2{1.5}}\]
				\item Option 2
				    \[ D =\frac{ \sqrt{15px\cdot 60px}}{2} \]
				    \[ S = \frac{D}{3} \]	
				    \[ IoD = \log_2{\left(\frac{D}{\frac{D}{3}} + 1 \right)} = \log_2{4} \]
				\item Option 3
					\[ D =\frac{ \sqrt{25px\cdot 60px}}{2} \]
					\[ S = \frac{D}{5} \]
					\[ IoD = \log_2{\left(\frac{D}{\frac{D}{5}} + 1 \right)} = \log_2{6} \]
				\item Option 4
					\[ D =\frac{ \sqrt{35px\cdot 60px}}{2} \]
					\[ S = \frac{D}{7} \]
					\[ IoD = \log_2{\left(\frac{D}{\frac{D}{7}} + 1 \right)} = \log_2{8} \]
			\end{enumerate}
				
			\item Pie-Menü\\
				Für das Pie-Menü sind die Index of Difficulties für alle vier Optionen gleich:
				\[ D = 20px \]
				\[ S = 40px = 2D \]
				\[ IoD = \log_2{\left(\frac{D}{2D} + 1 \right) = \log_2{1.5}}\]
				
			
		\end{enumerate}
	\item
		Für das Pie-Menü ist der Index of Difficultiy für jede Option gleich, es werden also alle Optionnen gleich schnell erreicht.
		Dies ist für das Pull-Down-Menü nicht der Fall, je weiter die Option vom Zeiger entfernt liegt, desto höher ist der Index of Difficulty - dieser wächst logarithmisch. 
		
		Das Pie-Menü bietet sich also z.B. als Kontextmenü an, wenn es nicht am Rande des Bildschirms liegt (bzw müsste hier ein Halb- oder Viertelkreis verwendet werden).
		Außerdem dürfen nicht zu viele Optionen vorhanden sein, da die klickbare Fläche mit jeder zusätzichen Option relativ weniger wird.
		Auch eine Verschachtelung ist möglich, wird aber ggf. durch Überdeckung schnell unübersichtlich.
\end{enumerate}



\end{document}
