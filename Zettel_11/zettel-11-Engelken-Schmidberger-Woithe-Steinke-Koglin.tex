\documentclass[a4paper,10pt]{article}

% Hier die Nummer des Blatts und Autoren angeben.
\newcommand{\blatt}{11}
\newcommand{\autor}{Ralf Engelken, Joachim Schmidberger, Frank Woithe, Michael Steinke, Merlin Koglin}

\usepackage{hci}
\usepackage{float} 

\begin{document}
% Seitenkopf mit Informationen
\kopf
\renewcommand{\figurename}{Figure}

\aufgabe{16 \textit{(Team-Aufgabe)}} 
Interview

\begin{enumerate}
\item{Login-Seite}

Analyse:
-Es gibt nacheinander 2 Login-Screens (Uni auswählen/ Benutzerlogin) von verschiedenen Servern Besser: 1 Login-Screen

-Login Seite 1:
-Die Informationen für UHH und Hamburger FH sind vermischt. Besser : Trennen
-Andere Informationen sind unübersichtlich angeordnet
-Untermenü: Es gibt keinen Menüpunkt für den Login. Wenn man einen Menüpunkt anwählt, kommt man nicht zur Login-Seite zurück.
-Nur weiterleitung zu Login-Seiten. Besser: direkter Login

-Login Seite 2:
-Umleitung auf RRZ-Seite mit anderem Design
-viele unnötige Informationen (wozu Single-Signon wenn man es nur für OLAT nutzen kann?)
Besser: Seite durch direkt-Login auf der ersten Seite ersetzen



\item{Willkommes-Seite}

Analyse:
-3 Menüebenen (horizontale Uni-Haupt-Menüleiste, vertikale Untermenüleiste, dann in Tabs aufgeteilte Seiten)
 Die Struktur ist sehr unübersichtlich, es ist schwierig sich zu orientieren. Besser: Tabs in Untermenü integrieren

-konfigurierbares Dashboard. Man kann Informationsfenster ein- oder ausblenden und in einem Grid mit 2 Spalten und dynamischer Zeilenzahl anordnen.
 Man kann die Größe der Fenster allerdings nicht beeinflussen. Manche sind fast leer, in andere passen nicht alle Informationen hinein.
 
-kritischer Fehler: Logout funktioniert nicht! Beim Logout werden die Session Cookies auf dem Server anscheinend nicht gelöscht, man kann sich nur dann nicht    mehr einloggen, wenn man die Cookies im Browser löscht!
\end{enumerate}
\end{document}
